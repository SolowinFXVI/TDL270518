\documentclass[a4paper]{article}
\usepackage{tikz}
\usetikzlibrary{automata,positioning,arrows}
\usepackage[T1]{fontenc}
\usepackage[utf8]{inputenc}
\usepackage{lmodern}
\usepackage[french]{babel}
\usepackage{listings}
\usepackage{color}
\usepackage{amsmath}
\usepackage{pifont}

\title{Compte - rendu : Evaluation d'expressions arithmétiques}
\author{JACQUET Julien 21400579}

\begin{document}
  \pagenumbering{gobble} %desactive la numerotation de page pour la page de garde
  \maketitle
  \newpage
  \pagenumbering{arabic} %reactive la numerotation

\paragraph{Rappel :}
Nous lisons notre équation arithmétique de droite à gauche.
\section*{Question 1}
  Soit une grammaire $G$ définie par $G=(\Sigma,V,S,P)$ telle que : \newline
  $\Sigma = {\{0,1,2,3,4,5,6,7,8,9,+,-,*,/,(,)\}}$ \newline
  $V = {\{S\}}$ \newline
  $S = {\{S\}}$
  \begin{flalign*} %alignement a gauche de la liste de regles de production
      & P = \{ S \rightarrow ( S ) ,\\
      & S \rightarrow \varepsilon ,\\
      & S \rightarrow S + S ,\\
      & S \rightarrow S - S ,\\
      & S \rightarrow S * S ,\\
      & S \rightarrow S / S ,\\
      & S \rightarrow n \} &
  \end{flalign*}
  $n$ = est un nombre quelconque.
\section*{Question 3:}
Etant donné la nature des symboles donnés pour notre langage, il va faut renommer certains éléments de notre langage :
\begin{flalign*}
&s = {\{+,*,/\}}\\
&m = {\{-\}}\\
&p_o ={\{(\}}\\
&p_f ={\{)\}}\\
&n = nombre\\
&L = {\{(p_o\ (m+\varepsilon)(s+\varepsilon)\ n\ (m+\varepsilon)\ s\ n\ p_f)\* + ((m+\varepsilon)(s+\varepsilon)\ n\ (m+\varepsilon)\ s\ n)\* + n\ \}} &
\end{flalign*}
note : Une fois une partie de l'equation evaluée elle devient un $n$ rendant le parenthèsage "récursif". Le $n$ à la fin du langage est présent pour éviter que notre équation soit vide.
\paragraph{}
On obtient l'automate suivant, avec une pile. \newline
$\Gamma={\{x\}}$\newline
$\delta$ symbole pile vide.\newline
$@$ peut importe ce qu'il y a dans la pile (x>=0).\newline
La reconnaissance se fait par pile vide et état final.\newline
On suppose que la pile est vide en $q_0$.\newline
Il est impossible de pop si la pile est vide $\rightarrow$ Rejet.\newline
Le $-$ dans la transition de $q_1$ à $q_6$ signifie que la pile est simplement lue mais pas changée.\newline
Enfin la pile n'est pas utilisée pour toutes les transitions.\newline
\hfill \break
\begin{tikzpicture}[>=stealth',shorten >=1pt,node distance=3.4cm,on grid,auto]
  \node[state,initial] (q_0) {$q_0$};
  \node[state] (q_1) [right=of q_0] {$q_1$};
  \node[state] (q_2) [right=of q_1] {$q_2$};
  \node[state] (q_3) [above right=of q_1] {$q_3$};
  \node[state] (q_4) [above left=of q_3] {$q_4$};
  \node[state] (q_5) [below right=of q_2] {$q_5$};
  \node[state,accepting] (q_6) [below left=of q_1] {$q_6$};
  \path[->]
  (q_0) edge node {n} (q_1)
        edge [loop above] node {$@,),\uparrow x$} ()
  (q_1) edge [bend left = 10] node {$+,-,*,/$} (q_2)
        edge [bend right = 10] node {$-$} (q_3)
        edge [bend right = 40] node {$\delta,\varepsilon,-$} (q_6)
        edge [loop below] node {$@,(,\downarrow$} ()
  (q_2) edge node {@,),$\uparrow x$} (q_5)
        edge [bend left=10] node {$n$} (q_1)
  (q_3) edge node {@,(,$\downarrow$} (q_4)
        edge [bend left] node {$+,/,*$} (q_2)
        edge [bend right = 60] node {$\varepsilon$} (q_1)
  (q_4) edge [bend right] node {$n,\varepsilon$} (q_1)
  (q_5) edge node {$n$} (q_1);

\end{tikzpicture}




\end{document}
