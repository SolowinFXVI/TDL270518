\documentclass[a4paper]{article}
\usepackage[T1]{fontenc}
\usepackage[utf8]{inputenc}
\usepackage{lmodern}
\usepackage[french]{babel}
\usepackage{listings}
\usepackage{color}
\usepackage{amsmath}
\usepackage{pifont}

\title{Compte - rendu : Evaluation d'expressions arithmétiques}
\author{JACQUET Julien 21400579}

\begin{document}
  \pagenumbering{gobble} %desactive la numerotation de page pour la page de garde
  \maketitle
  \newpage
  \pagenumbering{arabic} %reactive la numerotation

\section{Formalisation du Problème}
  \paragraph{Question 1}
  Soit une grammaire $G$ définie par $G=(\Sigma,V,S,P)$ telle que : \newline
  $\Sigma = {\{0,1,2,3,4,5,6,7,8,9,+,-,*,/,(,)\}}$ \newline
  $V = {\{S,X,Y\}}$ \newline
  $S = {\{S\}}$
  \begin{flalign*} %alignement a gauche de la liste de regles de production
      P = \{ \\
      & S \rightarrow X ,\\
      & S \rightarrow Y ,\\
      & X \rightarrow ( S ) ,\\
      & X \rightarrow S ,\\
      & X \rightarrow \varepsilon ,\\
      & Y \rightarrow S + n ,\\
      & Y \rightarrow S - n ,\\
      & Y \rightarrow S * n ,\\
      & Y \rightarrow S / n ,\\
      & Y \rightarrow S + S ,\\
      & Y \rightarrow S - S ,\\
      & Y \rightarrow S * S ,\\
      & Y \rightarrow S / S ,\\
      & Y \rightarrow n \} &
  \end{flalign*}
  $n$ = n'importe quelle combinaison contigüe  de $\{0,1,2,3,4,5,6,7,8,9\}$ (exemples : 3,45,1000,7845,65535,etc). \\




\end{document}
